% Options for packages loaded elsewhere
\PassOptionsToPackage{unicode}{hyperref}
\PassOptionsToPackage{hyphens}{url}
%
\documentclass[
]{article}
\usepackage{amsmath,amssymb}
\usepackage{lmodern}
\usepackage{iftex}
\ifPDFTeX
  \usepackage[T1]{fontenc}
  \usepackage[utf8]{inputenc}
  \usepackage{textcomp} % provide euro and other symbols
\else % if luatex or xetex
  \usepackage{unicode-math}
  \defaultfontfeatures{Scale=MatchLowercase}
  \defaultfontfeatures[\rmfamily]{Ligatures=TeX,Scale=1}
\fi
% Use upquote if available, for straight quotes in verbatim environments
\IfFileExists{upquote.sty}{\usepackage{upquote}}{}
\IfFileExists{microtype.sty}{% use microtype if available
  \usepackage[]{microtype}
  \UseMicrotypeSet[protrusion]{basicmath} % disable protrusion for tt fonts
}{}
\makeatletter
\@ifundefined{KOMAClassName}{% if non-KOMA class
  \IfFileExists{parskip.sty}{%
    \usepackage{parskip}
  }{% else
    \setlength{\parindent}{0pt}
    \setlength{\parskip}{6pt plus 2pt minus 1pt}}
}{% if KOMA class
  \KOMAoptions{parskip=half}}
\makeatother
\usepackage{xcolor}
\usepackage{color}
\usepackage{fancyvrb}
\newcommand{\VerbBar}{|}
\newcommand{\VERB}{\Verb[commandchars=\\\{\}]}
\DefineVerbatimEnvironment{Highlighting}{Verbatim}{commandchars=\\\{\}}
% Add ',fontsize=\small' for more characters per line
\newenvironment{Shaded}{}{}
\newcommand{\AlertTok}[1]{\textcolor[rgb]{1.00,0.00,0.00}{\textbf{#1}}}
\newcommand{\AnnotationTok}[1]{\textcolor[rgb]{0.38,0.63,0.69}{\textbf{\textit{#1}}}}
\newcommand{\AttributeTok}[1]{\textcolor[rgb]{0.49,0.56,0.16}{#1}}
\newcommand{\BaseNTok}[1]{\textcolor[rgb]{0.25,0.63,0.44}{#1}}
\newcommand{\BuiltInTok}[1]{\textcolor[rgb]{0.00,0.50,0.00}{#1}}
\newcommand{\CharTok}[1]{\textcolor[rgb]{0.25,0.44,0.63}{#1}}
\newcommand{\CommentTok}[1]{\textcolor[rgb]{0.38,0.63,0.69}{\textit{#1}}}
\newcommand{\CommentVarTok}[1]{\textcolor[rgb]{0.38,0.63,0.69}{\textbf{\textit{#1}}}}
\newcommand{\ConstantTok}[1]{\textcolor[rgb]{0.53,0.00,0.00}{#1}}
\newcommand{\ControlFlowTok}[1]{\textcolor[rgb]{0.00,0.44,0.13}{\textbf{#1}}}
\newcommand{\DataTypeTok}[1]{\textcolor[rgb]{0.56,0.13,0.00}{#1}}
\newcommand{\DecValTok}[1]{\textcolor[rgb]{0.25,0.63,0.44}{#1}}
\newcommand{\DocumentationTok}[1]{\textcolor[rgb]{0.73,0.13,0.13}{\textit{#1}}}
\newcommand{\ErrorTok}[1]{\textcolor[rgb]{1.00,0.00,0.00}{\textbf{#1}}}
\newcommand{\ExtensionTok}[1]{#1}
\newcommand{\FloatTok}[1]{\textcolor[rgb]{0.25,0.63,0.44}{#1}}
\newcommand{\FunctionTok}[1]{\textcolor[rgb]{0.02,0.16,0.49}{#1}}
\newcommand{\ImportTok}[1]{\textcolor[rgb]{0.00,0.50,0.00}{\textbf{#1}}}
\newcommand{\InformationTok}[1]{\textcolor[rgb]{0.38,0.63,0.69}{\textbf{\textit{#1}}}}
\newcommand{\KeywordTok}[1]{\textcolor[rgb]{0.00,0.44,0.13}{\textbf{#1}}}
\newcommand{\NormalTok}[1]{#1}
\newcommand{\OperatorTok}[1]{\textcolor[rgb]{0.40,0.40,0.40}{#1}}
\newcommand{\OtherTok}[1]{\textcolor[rgb]{0.00,0.44,0.13}{#1}}
\newcommand{\PreprocessorTok}[1]{\textcolor[rgb]{0.74,0.48,0.00}{#1}}
\newcommand{\RegionMarkerTok}[1]{#1}
\newcommand{\SpecialCharTok}[1]{\textcolor[rgb]{0.25,0.44,0.63}{#1}}
\newcommand{\SpecialStringTok}[1]{\textcolor[rgb]{0.73,0.40,0.53}{#1}}
\newcommand{\StringTok}[1]{\textcolor[rgb]{0.25,0.44,0.63}{#1}}
\newcommand{\VariableTok}[1]{\textcolor[rgb]{0.10,0.09,0.49}{#1}}
\newcommand{\VerbatimStringTok}[1]{\textcolor[rgb]{0.25,0.44,0.63}{#1}}
\newcommand{\WarningTok}[1]{\textcolor[rgb]{0.38,0.63,0.69}{\textbf{\textit{#1}}}}
\usepackage{longtable,booktabs,array}
\usepackage{calc} % for calculating minipage widths
% Correct order of tables after \paragraph or \subparagraph
\usepackage{etoolbox}
\makeatletter
\patchcmd\longtable{\par}{\if@noskipsec\mbox{}\fi\par}{}{}
\makeatother
% Allow footnotes in longtable head/foot
\IfFileExists{footnotehyper.sty}{\usepackage{footnotehyper}}{\usepackage{footnote}}
\makesavenoteenv{longtable}
\setlength{\emergencystretch}{3em} % prevent overfull lines
\providecommand{\tightlist}{%
  \setlength{\itemsep}{0pt}\setlength{\parskip}{0pt}}
\setcounter{secnumdepth}{-\maxdimen} % remove section numbering
\ifLuaTeX
  \usepackage{selnolig}  % disable illegal ligatures
\fi
\IfFileExists{bookmark.sty}{\usepackage{bookmark}}{\usepackage{hyperref}}
\IfFileExists{xurl.sty}{\usepackage{xurl}}{} % add URL line breaks if available
\urlstyle{same} % disable monospaced font for URLs
\hypersetup{
  hidelinks,
  pdfcreator={LaTeX via pandoc}}

\author{}
\date{}

\begin{document}

\hypertarget{gcc--gdb}{%
\section{GCC \& GDB}\label{gcc--gdb}}

gcc 与 g++ 分别是 gnu 的 c \& c++ 编译器 gcc/g++
在执行编译工作的时候,总共需要4步:

\begin{itemize}
\item
  1、预处理,生成 .i 的文件{[}\textbf{预处理器cpp}{]}
\item
  2、将预处理后的文件转换成汇编语言, 生成文件 .s
  {[}\textbf{编译器egcs}{]}
\item
  3、有汇编变为目标代码(机器代码)生成 .o 的文件{[}\textbf{汇编器as}{]}
\item
  4、连接目标代码, 生成可执行程序 {[}\textbf{链接器ld}{]}
\end{itemize}

\hypertarget{gcc}{%
\subsection{GCC}\label{gcc}}

\hypertarget{1ux8bbeux5b9aux6587ux4ef6ux6240ux4f7fux7528ux7684ux8bedux8a00--x}{%
\subsection{1、设定文件所使用的语言
-x}\label{1ux8bbeux5b9aux6587ux4ef6ux6240ux4f7fux7528ux7684ux8bedux8a00--x}}

可以编译不是指定文件后缀的文件,.c .cpp

\begin{Shaded}
\begin{Highlighting}[]
\NormalTok{gcc {-}x language filename}
\end{Highlighting}
\end{Shaded}

\begin{Shaded}
\begin{Highlighting}[]
\NormalTok{gcc {-}x none filename}
\end{Highlighting}
\end{Shaded}

\hypertarget{2ux53eaux628aux7a0bux5e8fux505aux6210objux6587ux4ef6--c}{%
\subsection{2、只把程序做成obj文件
-c}\label{2ux53eaux628aux7a0bux5e8fux505aux6210objux6587ux4ef6--c}}

\begin{Shaded}
\begin{Highlighting}[]
\NormalTok{gcc {-}c hello.c}
\end{Highlighting}
\end{Shaded}

\hypertarget{3ux53eaux6fc0ux6d3bux9884ux5904ux7406ux548cux7f16ux8bd1ux7f16ux8bd1ux4e3aux6c47ux7f16ux4ee3ux7801--s}{%
\subsection{3、只激活预处理和编译,编译为汇编代码
-S}\label{3ux53eaux6fc0ux6d3bux9884ux5904ux7406ux548cux7f16ux8bd1ux7f16ux8bd1ux4e3aux6c47ux7f16ux4ee3ux7801--s}}

\begin{Shaded}
\begin{Highlighting}[]
\NormalTok{gcc {-}S hello.c}
\end{Highlighting}
\end{Shaded}

\hypertarget{4ux53eaux6fc0ux6d3bux9884ux5904ux7406ux4e0dux751fux6210ux6587ux4ef6--e}{%
\subsection{4、只激活预处理,不生成文件
-E}\label{4ux53eaux6fc0ux6d3bux9884ux5904ux7406ux4e0dux751fux6210ux6587ux4ef6--e}}

\begin{Shaded}
\begin{Highlighting}[]
\NormalTok{gcc {-}E hello.c \textgreater{} text.txt
}
\NormalTok{gcc {-}E hello.c | more}
\end{Highlighting}
\end{Shaded}

\hypertarget{5ux6307ux5b9aux76eeux6807ux540dux79f0ux9ed8ux8ba4ux7f16ux8bd1ux51faaout--o}{%
\subsection{5、指定目标名称,默认编译出a.out
-o}\label{5ux6307ux5b9aux76eeux6807ux540dux79f0ux9ed8ux8ba4ux7f16ux8bd1ux51faaout--o}}

\begin{Shaded}
\begin{Highlighting}[]
\NormalTok{gcc {-}o ExecuteFileName hello.c}
\end{Highlighting}
\end{Shaded}

\hypertarget{6ux4f7fux7528ux7ba1ux9053ux4ee3ux66ffux7f16ux8bd1ux4e2dux4e34ux65f6ux6587ux4ef6--pipe}{%
\subsection{6、使用管道代替编译中临时文件
-pipe}\label{6ux4f7fux7528ux7ba1ux9053ux4ee3ux66ffux7f16ux8bd1ux4e2dux4e34ux65f6ux6587ux4ef6--pipe}}

使用非GNU汇编工具的时候,可以会有问题。

\begin{Shaded}
\begin{Highlighting}[]
\NormalTok{gcc {-}pipe {-}o hello.exe hello.c}
\end{Highlighting}
\end{Shaded}

\hypertarget{7ux4f7fux7528ux67d0ux4e2aux6587ux4ef6-include}{%
\subsection{7、使用某个文件,-include}\label{7ux4f7fux7528ux67d0ux4e2aux6587ux4ef6-include}}

相当于在代码中使用\#include

\begin{Shaded}
\begin{Highlighting}[]
\NormalTok{gcc hello.c {-}include /root/tmp.h}
\end{Highlighting}
\end{Shaded}

\hypertarget{8ux9884ux5904ux7406ux662fux4e0dux5220ux9664ux6ce8ux91caux4fe1ux606fux4e00ux822cux548c-eux4f7fux7528--c}{%
\subsection{8、预处理是不删除注释信息,一般和-E使用
-C}\label{8ux9884ux5904ux7406ux662fux4e0dux5220ux9664ux6ce8ux91caux4fe1ux606fux4e00ux822cux548c-eux4f7fux7528--c}}

\hypertarget{9ux751fux6210ux6587ux4ef6ux5173ux8054ux7684ux4fe1ux606f--m}{%
\subsection{9、生成文件关联的信息
-M}\label{9ux751fux6210ux6587ux4ef6ux5173ux8054ux7684ux4fe1ux606f--m}}

和-M的那个一样,但是它将忽略由 \textbf{\#include} 造成的依赖关系。

\begin{Shaded}
\begin{Highlighting}[]
\NormalTok{{-}MD}
\end{Highlighting}
\end{Shaded}

和-M相同,但是输出将导入到.d的文件里面

\begin{Shaded}
\begin{Highlighting}[]
\NormalTok{{-}MMD}
\end{Highlighting}
\end{Shaded}

和 -MM 相同,但是输出将导入到 .d 的文件里面。

\hypertarget{10ux5236ux5b9aux7f16ux8bd1ux7684ux65f6ux5019ux4f7fux7528ux7684ux5e93--llibrary}{%
\subsection{\texorpdfstring{10、制定编译的时候使用的库
\textbf{-llibrary}}{10、制定编译的时候使用的库 -llibrary}}\label{10ux5236ux5b9aux7f16ux8bd1ux7684ux65f6ux5019ux4f7fux7528ux7684ux5e93--llibrary}}

例子用法

\begin{Shaded}
\begin{Highlighting}[]
\NormalTok{gcc {-}lcurses hello.c}
\end{Highlighting}
\end{Shaded}

使用 ncurses 库编译程序

\hypertarget{gdb}{%
\subsection{GDB}\label{gdb}}

\hypertarget{1-ux5165ux95e8ux4e09ux6b65ux66f2}{%
\subsection{1. 入门三步曲}\label{1-ux5165ux95e8ux4e09ux6b65ux66f2}}

\hypertarget{11-ux8bbeux65adux70b9}{%
\subsubsection{1.1 设断点}\label{11-ux8bbeux65adux70b9}}

如果你曾经使用过调试器, 那你可能已经会设置断点了。
下面是一个我们要调试的程序(虽然没有任何 Bug):

\begin{Shaded}
\begin{Highlighting}[]
\PreprocessorTok{\#include }\ImportTok{\textless{}stdio.h\textgreater{}}\PreprocessorTok{
}
\DataTypeTok{void}\NormalTok{ do\_thing}\OperatorTok{()} \OperatorTok{\{}

\NormalTok{ printf}\OperatorTok{(}\StringTok{"Hi!}\SpecialCharTok{\textbackslash{}n}\StringTok{"}\OperatorTok{);}

\OperatorTok{\}}

\DataTypeTok{int}\NormalTok{ main}\OperatorTok{()} \OperatorTok{\{}

\NormalTok{ do\_thing}\OperatorTok{();}

\OperatorTok{\}}
\end{Highlighting}
\end{Shaded}

另存为 hello.c. 我们可以使用 dbg 调试它, 像这样:

\begin{Shaded}
\begin{Highlighting}[]
\NormalTok{bork@kiwi \textasciitilde{}\textgreater{} gcc {-}g hello.c {-}o hello
}
\NormalTok{bork@kiwi \textasciitilde{}\textgreater{} gdb ./hello}
\end{Highlighting}
\end{Shaded}

以上是带调试信息编译 hello.c(为了 gdb 可以更好工作),
并且它会给我们醒目的提示符, 就像这样: \texttt{(gdb)} 我们可以使用 break
命令设置断点, 然后使用 run 开始调试程序。

\begin{Shaded}
\begin{Highlighting}[]
\NormalTok{(gdb) break do\_thing 
}
\NormalTok{Breakpoint 1 at 0x4004f8
}
\NormalTok{(gdb) run
}
\NormalTok{Starting program: /home/bork/hello 
}
\NormalTok{Breakpoint 1, 0x00000000004004f8 in do\_thing ()}
\end{Highlighting}
\end{Shaded}

程序暂停在了 do\_thing 开始的地方。 我们可以通过 where
查看我们所在的调用栈。

\begin{Shaded}
\begin{Highlighting}[]
\NormalTok{(gdb) where
}
\NormalTok{\#0 do\_thing () at hello.c:3
}
\NormalTok{\#1 0x08050cdb in main () at hello.c:6
}
\NormalTok{(gdb)}
\end{Highlighting}
\end{Shaded}

\hypertarget{12-ux9605ux8bfbux6c47ux7f16ux4ee3ux7801}{%
\subsubsection{1.2
阅读汇编代码}\label{12-ux9605ux8bfbux6c47ux7f16ux4ee3ux7801}}

使用 disassemble 命令, 我们可以看到这个函数的汇编代码。棒级了, 这是 x86
汇编代码。虽然我不是很懂它, 但是 callq 这一行是 printf 函数调用。

\begin{Shaded}
\begin{Highlighting}[]
\NormalTok{(gdb) disassemble do\_thing
}
\NormalTok{Dump of assembler code for function do\_thing:
}
\NormalTok{ 0x00000000004004f4 \textless{}+0\textgreater{}: push \%rbp
}
\NormalTok{ 0x00000000004004f5 \textless{}+1\textgreater{}: mov \%rsp,\%rbp
}
\NormalTok{=\textgreater{} 0x00000000004004f8 \textless{}+4\textgreater{}: mov $0x40060c,\%edi
}
\NormalTok{ 0x00000000004004fd \textless{}+9\textgreater{}: callq 0x4003f0 
}
\NormalTok{ 0x0000000000400502 \textless{}+14\textgreater{}: pop \%rbp
}
\NormalTok{ 0x0000000000400503 \textless{}+15\textgreater{}: retq}
\end{Highlighting}
\end{Shaded}

你也可以使用 disassemble 的缩写 disas。

\hypertarget{13-ux67e5ux770bux5185ux5b58}{%
\subsubsection{1.3 查看内存}\label{13-ux67e5ux770bux5185ux5b58}}

当调试我的内核时, 我使用 gdb 的主要原因是,
以确保内存布局是如我所想的那样。检查内存的命令是 examine, 或者使用缩写
x。我们将使用x。 通过阅读上面的汇编代码, 似乎 0x40060c
可能是我们所要打印的字符串地址。我们来试一下。

\begin{Shaded}
\begin{Highlighting}[]
\NormalTok{(gdb) x/s 0x40060c
}
\NormalTok{0x40060c: "Hi!"}
\end{Highlighting}
\end{Shaded}

的确是这样。x/s 中 /s 部分, 意思是"把它作为字符串展示"。我也可以"展示 10
个字符", 像这样:

\begin{Shaded}
\begin{Highlighting}[]
\NormalTok{(gdb) x/10c 0x40060c
}
\NormalTok{0x40060c: 72 \textquotesingle{}H\textquotesingle{} 105 \textquotesingle{}i\textquotesingle{} 33 \textquotesingle{}!\textquotesingle{} 0 \textquotesingle{}\textbackslash{}000\textquotesingle{} 1 \textquotesingle{}\textbackslash{}001\textquotesingle{} 27 \textquotesingle{}\textbackslash{}033\textquotesingle{} 3 \textquotesingle{}\textbackslash{}003\textquotesingle{} 59 \textquotesingle{};\textquotesingle{}
}
\NormalTok{0x400614: 52 \textquotesingle{}4\textquotesingle{} 0 \textquotesingle{}\textbackslash{}000\textquotesingle{}}
\end{Highlighting}
\end{Shaded}

你可以看到前四个字符是 H、i、! 和 \textbackslash0,
并且它们之后的是一些不相关的东西。

\hypertarget{2-gdbux53c2ux6570ux547dux4ee4-ux8bf4ux660e}{%
\subsection{2. gdb参数+命令
说明}\label{2-gdbux53c2ux6570ux547dux4ee4-ux8bf4ux660e}}

\hypertarget{21-gdb-ux53c2ux6570}{%
\subsubsection{2.1 gdb 参数}\label{21-gdb-ux53c2ux6570}}

-cd: 设置工作目录; -q: 安静模式, 不打印介绍信息和版本信息; -d:
添加文件查找路径; -x: 从指定文件中执行GDB指令; -s:
设置读取的符号表文件。

\hypertarget{22-gdb-ux547dux4ee4}{%
\subsubsection{2.2 gdb 命令}\label{22-gdb-ux547dux4ee4}}

\begin{longtable}[]{@{}lll@{}}
\toprule()
命令 & 解释 & 示例 \\
\midrule()
\endhead
file \textless 文本名\textgreater{} &
加载被调试的可执行程序文件。因为一般都在被调试程序所在目录下执行GDB,
因而文本名不需要带路径。 & (gdb) file gdb-sample \\
r & Run的简写, 运行被调试的程序。如果此前没有下过断点, 则执行完整个程序;
如果有断点, 则程序暂停在第一个可用断点处。 & (gdb) r \\
c & Continue的简写, 继续执行被调试程序, 直至下一个断点或程序结束。 &
(gdb) c \\
b \textless 行号\textgreater{} b \textless 函数名称\textgreater{} b
*\textless 函数名称\textgreater{} b *\textless 代码地址\textgreater{} d
{[}编号{]} & b: Breakpoint的简写, 设置断点。
可以使用"行号"``函数名称"``执行地址"等方式指定断点位置。
其中在函数名称前面加''*``符号表示将断点设置在"由编译器生成的prolog代码处''。如果不了解汇编,
可以不予理会此用法。 d: Delete breakpoint的简写, 删除指定编号的某个断点,
或删除所有断点。断点编号从1开始递增。 & (gdb) b 8 (gdb) b main (gdb) b
*main (gdb) b *0x804835c (gdb) d \\
s, n & s: 执行一行源程序代码, 如果此行代码中有函数调用, 则进入该函数; n:
执行一行源程序代码, 此行代码中的函数调用也一并执行。 s
相当于其它调试器中的"Step Into (单步跟踪进入)"; n
相当于其它调试器中的"Step Over (单步跟踪)"。
这两个命令必须在有源代码调试信息的情况下才可以使用(GCC编译时使用''-g"参数)。
& (gdb) s (gdb) n \\
si, ni & si命令类似于s命令, ni命令类似于n命令。所不同的是,
这两个命令(si/ni)所针对的是汇编指令, 而s/n针对的是源代码。 & (gdb) si
(gdb) ni \\
p \textless 变量名称\textgreater{} & Print的简写,
显示指定变量(临时变量或全局变量)的值。 & (gdb) p i (gdb) p nGlobalVar \\
display \ldots{} undisplay \textless 编号\textgreater{} & display,
设置程序中断后欲显示的数据及其格式。 例如,
如果希望每次程序中断后可以看到即将被执行的下一条汇编指令, 可以使用命令
``display /i \(pc” 其中 \)pc 代表当前汇编指令, /i
表示以十六进行显示。当需要关心汇编代码时, 此命令相当有用。 undispaly,
取消先前的display设置, 编号从1开始递增。 & (gdb) display /i \$pc (gdb)
undisplay 1 \\
i & info的简写, 用于显示各类信息, 详情请查阅"help i"。 & (gdb) i r \\
q & Quit的简写, 退出GDB调试环境。 & (gdb) q \\
help {[}命令名称{]} & GDB帮助命令, 提供对GDB名种命令的解释说明。
如果指定了"命令名称"参数, 则显示该命令的详细说明; 如果没有指定参数,
则分类显示所有GDB命令, 供用户进一步浏览和查询。 & (gdb) help \\
l & list的简写, 列出10行代码, 再次输入则列出往下的10行。若是list
加行号则从该行开始输出10行 & (gdb) l \\
watch & watch 为表达式(变量)expr设置一个观察点。变量量表达式值有变化时,
马上停住程序 & (gdb) watch a*b + c/d \\
finish & 运行至当前函数结束,
并打印函数返回时的堆栈地址和返回值及参数值等信息 & \\
set args & 可以修改发送给程序的参数 & (gdb)set args --b --x \\
show args & 查看其缺省参数的列表 & (gdb) show args \\
whatis & 显示某个变量的类型 & (gdb) whatis p \\
\bottomrule()
\end{longtable}

\hypertarget{23-ux673aux5668ux8bedux8a00ux5de5ux5177}{%
\subsubsection{2.3
机器语言工具}\label{23-ux673aux5668ux8bedux8a00ux5de5ux5177}}

有一组专用的gdb变量可以用来检查和修改计算机的通用寄存器,
gdb提供了目前每一台计算机中实际使用的4个寄存器的标准名字:

\begin{itemize}
\item
  l \$pc : 程序计数器
\item
  l \$fp : 帧指针(当前堆栈帧)
\item
  l \$sp : 栈指针
\item
  l \$ps : 处理器状态
\end{itemize}

\end{document}
